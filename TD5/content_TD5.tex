
\newcommand{\numTD}{TD5}
\newcommand{\themeTD}{Classifier des chansons}


\begin{center}
\begin{tabular}{|p{2cm}c|}
\hline
{\includegraphics[width=1.8cm,viewport=0 0 337 248]{../images/sorbonne.png}} & \raisebox{2ex}{\begin{Large}\textbf{M1SOL023 Méthodologie de la Recherche}\end{Large}}\\
2021-2022& \raisebox{2ex}{\begin{Large}\textbf{ en Langue et Informatique}\end{Large}}\\
&  \begin{large}\textbf{\numTD}\end{large} \\
&  \begin{large} \textbf{\themeTD}\end{large} \\
& Gaël Lejeune, Sorbonne Université \\
& \tiny{Inspiré de Agnès Delaborde 2015-2016}\\
\hline
\end{tabular}
\end{center}


\hrule
%%%%%%%%%%%%%%%%%%%%%%%%%EN-TETE%%%%%%%%%%%%%%%%%%%%%%%%%%%%%
%\renewcommand{\contentsname}{Sommaire du TD}
%\tableofcontents
%\newpage

\noindent\fcolorbox{red}{lightgray}{
\begin{minipage}{12cm}
\section*{Objectifs}

\begin{itemize}
  \item Parser un corpus de chansons en XML
  \item Représenter informatiquement les textes des chansons
  \item Apprendre à découvrir les auteurs
  \item Varier les représentations informatiques des textes
%  \item Maîtriser la terminologie : classes, instances, attributs \dots
\end{itemize}
\end{minipage}
}

\vspace{3cm}
Le code en ligne sur Moodle vous permet de manipuler le corpus et de faire un premier essai de classification.

Travail à commencer en TD et à finir pour la prochaine fois :

\begin{itemize}
  \item Modifier la manière de représenter les textes en epxloitant le sparamètres du CountVectorizer\footnote{\url{https://scikit-learn.org/stable/modules/generated/sklearn.feature_extraction.text.CountVectorizer.html}} :
  \begin{itemize}
    \item enlever les mots outils (\texttt{stopwords})
    \item prendre seulement les 1000 mots les plus fréquents du corpus (\texttt{max\_features})
    \item travailler sur des n-grammes de mots (\texttt{ngram\_range})
    \item travailler sur des n-grammes de caractères (\texttt{analyzer})
  \end{itemize}
  \item Noter quand les résultats évoluent positivement ou pas
  \item Combiner ces pré-traitements
  \item Tester 2 autres classifieurs, comparer leurs résultats
  \item Réfléchissez à des améliorations possibles sur la vectorisation des textes (types d'observables, de pré-taritements, de filtres \dots)
\end{itemize}

Déposez sur Moodle pour le 24/10 votre code + 1 page de commentaire des résultats et possibilités d'améliorations (en PDF)
