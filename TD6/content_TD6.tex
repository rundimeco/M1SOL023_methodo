
\newcommand{\numTD}{TD6}
\newcommand{\themeTD}{Aller plus loin avec \textsc{SciKitlearn}}


\begin{center}
\begin{tabular}{|p{2cm}c|}
\hline
{\includegraphics[width=1.8cm,viewport=0 0 337 248]{../images/sorbonne.png}} & \raisebox{2ex}{\begin{Large}\textbf{M1SOL023 Méthodologie de la Recherche}\end{Large}}\\
2021-2022& \raisebox{2ex}{\begin{Large}\textbf{ en Langue et Informatique}\end{Large}}\\
&  \begin{large}\textbf{\numTD}\end{large} \\
&  \begin{large} \textbf{\themeTD}\end{large} \\
& Gaël Lejeune, Sorbonne Université \\
& \tiny{Inspiré de Agnès Delaborde 2015-2016}\\
\hline
\end{tabular}
\end{center}


\hrule
%%%%%%%%%%%%%%%%%%%%%%%%%EN-TETE%%%%%%%%%%%%%%%%%%%%%%%%%%%%%
%\renewcommand{\contentsname}{Sommaire du TD}
%\tableofcontents
%\newpage

\noindent\fcolorbox{red}{lightgray}{
\begin{minipage}{12cm}
\section*{Objectifs}

\begin{itemize}
  \item Utiliser classification report
  \item Comprendre les paramètres des classifieurs
  \item Tester de nouvelles caractéristiques
  \item Combiner des jeux de caractéristiques
\end{itemize}
\end{minipage}
}
\vspace{.5cm}

Le code ci-dessous est donné dans le notebook Python sur Moodle.

\section{Retour sur ce que l'on a fait précédemment}

Notez bien que vous ne trouverez pas systématiquement les mêmes résultats que moi du fait du découpage train/test, l'ordre de grandeur ne doit pas bouger par contre.

 \subsection*{Récupérer les instances (X) et les classes (y) et vectoriser}

\begin{python}
import pandas as pd
data_spam =pd.read_csv("spamham.csv")
y = data_spam["spam"]
from sklearn.feature_extraction.text import CountVectorizer
V = CountVectorizer(ngram_range = (1,2) )
X = V.fit_transform(data_spam["text"])
\end{python}


\subsection*{Séparer train test}
\begin{python}
from sklearn.model_selection import train_test_split
X_train, X_test, y_train, y_test = train_test_split(X, y, test_size=0.3, random_state=0)
\end{python}

\subsection*{Récupérer les instances (X) et les classes (y) et vectoriser}
\begin{python}
y = data_spam["spam"]
from sklearn.feature_extraction.text import CountVectorizer
V = CountVectorizer(ngram_range = (1,2) )
X = V.fit_transform(data_spam["text"])
\end{python}

\subsection*{Séparer train et test}

\begin{python}
from sklearn.model_selection import train_test_split
X_train, X_test, y_train, y_test = train_test_split(X, y, test_size=0.3, random_state=0)
\end{python}

\subsection*{Classifier et évaluer}


\begin{python}
from sklearn.linear_model import Perceptron
ppn = Perceptron(eta0=0.1, random_state=0)
ppn.fit(X_train, y_train)
y_pred = ppn.predict(X_test)
\end{python}

Avec un exemple simple d'évaluation, l'exactitude, où on fait la somme de tous les cas où la valeur dans y\_test est bien trouvée dans y\_pred :

\begin{python}
print('Bons résultats %d' % (y_test == y_pred).sum())
print('Erreurs: %d' % (y_test != y_pred).sum())
\end{python}

Ce qui nous donnait:
\begin{itemize}
  \item Bons résultats 1687
  \item Erreurs: 32
\end{itemize}

\section{Pour aller plus loin dans l'évaluation}

Le module classification\_report de \textsc{SciKitLearn}, code ci-dessous, nous eprmet d'obtenir le résultat figurant dans le tableau \ref{class_report1}.

\begin{python}
from sklearn.metrics import classification_report
report = classification_report(y_test, y_pred)
print(report)
\end{python}

\begin{table}[h]
\begin{center}
\begin{tabular}{lllll}
\hline
 &   precision& recall & f1-score &  support\\
\hline

    0&    0.99&   0.98&   0.99&   1314\\
    1&    0.95&   0.98&   0.96&    405\\
\hline

   micro avg&    0.98&   0.98&   0.98&   1719\\
   macro avg&    0.97&   0.98&   0.97&   1719\\
weighted avg&    0.98&   0.98&   0.98&   1719\\
\hline
\end{tabular}
\caption{Rapport de classification "de base"\label{class_report1}}
\end{center}
\end{table}

Comme nous savons que le 1 c'est spam et le 0 c'est ham, on peut écrire ceci :

\begin{python}
nom_classes = ["ham", "spam"]
report = classification_report(y_test, y_pred, target_names=nom_classes)
print(report)
\end{python}

Ce qui donne le même tableau mais avec cette fois les noms des classes ou \textit{labels} (cf. Tableau \ref{class_report2}). 


\begin{table}
\begin{center}
\begin{tabular}{lllll}

\hline
 &   precision& recall&  f1-score&   support\\
\hline

  ham&    0.99&   0.98&   0.99&   1314\\
 spam&    0.95&   0.98&   0.96&    405\\
\hline

   micro avg&    0.98&   0.98&   0.98&   1719\\
   macro avg&    0.97&   0.98&   0.97&   1719\\
weighted avg&    0.98&   0.98&   0.98&   1719\\

\end{tabular}
\caption{Rapport de classification avec les noms des classes\label{class_report2}}
\end{center}
\end{table}

Un peu de terminologie :

\begin{itemize}
  \item Précision, rappel, f-mesure: c'est du déjà vu
  \item Le support c'est le nombre d'instances concernées
  \item La micro f-mesure c'est la moyenne \textbf{pondérée} des F-mesure (une classe compte en fonction de sa taille), ça présente mieux la classe majoritaire
  \item La macro f-mesure c'est la moyenne des F-mesure de chaque classe (idnépendamment de leur taille), ça représente mieux la classe minoritaire généralement plus difficile à trouver
\end{itemize}

 \subsection*{ On peut enregistrer le classification report pour s'en servir plus tard}

\begin{python}
w = open("report_classifier=perceptron_dataset=spam.txt", "w")
w.write(report)
w.close()
\end{python}

 \subsection*{On peut avoir les résultats sous forme de structure de données Python}

\begin{python}
stats = precision_recall_fscore_support(y_test, y_pred)
print(stats)
from sklearn.metrics import precision_recall_fscore_support
stats = precision_recall_fscore_support(y_test, y_pred)
print(stats)
\end{python}

\subsection*{Dans l'ordre les précisions pour chaque classe, puis les rappels, les F-mesure et les supports}

\begin{python}
( array([0.99307692, 0.9451074 ]), 
  array([0.98249619, 0.97777778]), 
  array([0.98775822, 0.96116505]), 
  array([1314,  405]))
\end{python}

\subsection*{Et maintenant la matrice de confusion}

\begin{python}
print(matrice_confusion)
from sklearn.metrics import confusion_matrix
matrice_confusion = confusion_matrix(y_test, y_pred)
print(matrice_confusion)
\end{python}

\begin{table}
\begin{center}
\begin{tabular}{ll}
1291   &23 \\
   9   &396\\
\end{tabular}
\end{center}
\caption{Matrice de Confusion}
\end{table}

\section{Explorer les paramètres du classifieur arbre de décision} % pour jouer avec les paramètres

\begin{python}
from sklearn import tree
DT = tree.DecisionTreeClassifier()
DT = DT.fit(X_train, y_train)
y_pred = DT.predict(X_test)

matrice_confusion = confusion_matrix(y_test, y_pred)
print(matrice_confusion)
\end{python}

Résultat :

\begin{python}
[[1257   57]
 [  32  373]]
\end{python}


Regardons l'impact du paramètres \texttt{random\_state}

\begin{python}
print("Avec la valeur par défaut de random state")
for i in range(3):
  DT = tree.DecisionTreeClassifier()
  DT = DT.fit(X_train, y_train)
  y_pred = DT.predict(X_test)
  matrice_confusion = confusion_matrix(y_test, y_pred)
  print(matrice_confusion)
  stats = precision_recall_fscore_support(y_test, y_pred)
  print(stats)

\end{python}

On voit que les résultats varient entre les itérations :

\begin{python}
[[1249   65]
 [  29  376]]
(array([0.97730829, 0.85260771]), 
 array([0.95053272, 0.92839506]), 
 array([0.96373457, 0.88888889]), 
 array([1314,  405]))
--------------------
[[1253   61]
 [  35  370]]
(array([0.97282609, 0.85846868]), 
 array([0.95357686, 0.91358025]), 
 array([0.9631053 , 0.88516746]), 
 array([1314,  405]))
--------------------
[[1255   59]
 [  35  370]]
(array([0.97286822, 0.86247086]), 
 array([0.95509893, 0.91358025]), 
 array([0.96390169, 0.88729017]), 
 array([1314,  405]))
--------------------
\end{python}


 En fixant random state ça ne bouge plus : 

\begin{python}
for i in range(3):
  DT = tree.DecisionTreeClassifier(random_state=0)
  DT = DT.fit(X_train, y_train)
  y_pred = DT.predict(X_test)
  matrice_confusion = confusion_matrix(y_test, y_pred)
  print(matrice_confusion)
  stats = precision_recall_fscore_support(y_test, y_pred)
  print(stats)
  print("--"*10)
\end{python}

Allez voir l'explication donnée sur ce paramètre \texttt{random\_state} ici :

\small{\url{https://scikit-learn.org/stable/modules/generated/sklearn.tree.DecisionTreeClassifier.html}}

Intéressez vous ensuite aux paramètres suivants: \texttt{max\_depth}, \texttt{min\_samples\_split}, \texttt{min\_samples\_leaf} et \texttt{max\_features}.
Lancez différentes expériences pour voir leur impact respectif.

\section{Avec d'autres caractéristiques}

Ce que l'on a fait jusqu'ici c'est du sac de mots, ou Bag of Words (BOW), même si on ajouté des choses:

\begin{itemize}
  \item n-grams
  \item caractères plutôt que mots (sac de caractères si on peut dire)
  \item pondération tf-idf
\end{itemize}

On va tester d'autres caractéristiques maintenant, ce que l'on va appeler des caractéristiques "stylistiques". On va pour chaque texte compter quelques caractéristiques, les ranger dans un vecteur et donner ça au classifieur.

\begin{python}
import statistics

X_stylo = []#notre nouvelle matrice de description
for text in data_spam["text"]:
    liste_mots = text.split()
    phrases = text.split(". ")
    NB_phrases = len(phrases)
    NB_mots = len(liste_mots)
    NB_caracteres = len(text)
    moyenne_taille_mots = statistics.mean([len(x) for x in liste_mots])
    moyenne_taille_phrases = NB_mots/NB_phrases
    caracteristiques = [NB_phrases, NB_mots, NB_caracteres, moyenne_taille_mots, moyenne_taille_phrases]
    X_stylo.append(caracteristiques)
\end{python}

Chaque texte est décrit par 5 caractéristiques, et on va voir si le classifieur arrive à quelque chose.

 Notez bien dans les parenthèses que l'on utilise X\_stylo désormais

\begin{python}
X_train, X_test, y_train, y_test = train_test_split(X_stylo, y, test_size=0.3, random_state=0)
DT = tree.DecisionTreeClassifier()
DT = DT.fit(X_train, y_train)
y_pred = DT.predict(X_test)
matrice_confusion = confusion_matrix(y_test, y_pred)
print(matrice_confusion)
stats = precision_recall_fscore_support(y_test, y_pred)
print(stats)
\end{python}

Ce qui donne :

\begin{python}
[[1097  217]
 [ 196  209]]
(array([0.84841454, 0.49061033]), 
 array([0.8348554 , 0.51604938]), 
 array([0.84158036, 0.50300842]), 
 array([1314,  405]))
\end{python}

\begin{table}[h]
\begin{center}
\begin{tabular}{lllll}
\hline
 &   precision& recall & f1-score &  support\\
\hline
	 ham&   0.85&  0.83&  0.84&  1314\\
	spam&   0.49&  0.52&  0.50&   405\\
\hline
   micro avg&   0.76&  0.76&  0.76&  1719\\
   macro avg&   0.67&  0.68&  0.67&  1719\\
weighted avg&   0.76&  0.76&  0.76&  1719\\
\hline
\end{tabular}
\caption{Classifieur stylométrique}
\end{center}
\end{table}


Verdict, ce n'est pas exceptionnel mais ça donne des résultats meilleur que si on mettait tout en spam ou tout en ham (ce que l'on appelle un classifieur \textsc{ZeroRule}:

\begin{python}
y_spam = [1 for x in y_test]##une liste remplie de 1
report = classification_report(y_spam, y_pred, target_names=nom_classes)
matrice_confusion = confusion_matrix(y_spam, y_pred)
print(matrice_confusion)
print(report)

y_ham = [0 for x in y_test]##une liste remplie de 0
report = classification_report(y_ham, y_pred, target_names=nom_classes)
matrice_confusion = confusion_matrix(y_ham, y_pred)
print(matrice_confusion)
print(report)
\end{python}

\begin{python}
[[   0 1314]#0 ham, Tout est spam
 [   0  405]]
\end{python}

\begin{table}[h]
\begin{center}
\begin{tabular}{lllll}
\hline
 &   precision& recall & f1-score &  support\\
\hline
	 ham&   0.00&  0.00&  0.00&  1314\\
	spam&   0.24&  1.00&  0.38&   405\\
\hline
   micro avg&   0.24&  0.24&  0.24&  1719\\
   macro avg&   0.12&  0.50&  0.19&  1719\\
weighted avg&   0.06&  0.24&  0.09&  1719\\
\hline
\end{tabular}
\caption{Classifieur \textsc{ZeroRule}: tout est \textit{spam}}
\end{center}
\end{table}

Puis en mettant tout en ham :

\begin{python}
[[1314    0]
 [ 405    0]]
\end{python}

\begin{table}[h]
\begin{center}
\begin{tabular}{lllll}
\hline
 &   precision& recall & f1-score &  support\\
\hline
 ham&   0.76&  1.00&  0.87&  1314\\
spam&   0.00&  0.00&  0.00&   405\\
\hline
   micro avg&   0.76&  0.76&  0.76&  1719\\
   macro avg&   0.38&  0.50&  0.43&  1719\\
weighted avg&   0.58&  0.76&  0.66&  1719\\


\hline
\end{tabular}
\caption{Classifieur \textsc{ZeroRule}: tout est \textit{ham}}
\end{center}
\end{table}

Observez bien la colonne rappel, remarquez que la colonne f-score montre que notre classifieur stylo marche correctement.

\subsection*{On peut combiner les deux ?}

Oui, on va concaténer des matrices :

\begin{python}
## on regarde la "forme" de X
print(X.shape[0])#NB lignes   -> instances
print(X.shape[1])#Nb colonnes -> caractéristiques

##on crée une sparse matrix avec notre X_stylo
from scipy.sparse import csr_matrix
sparse_stylo = csr_matrix(X_stylo)
print(sparse_stylo.shape[0])#NB lignes   -> instances
print(sparse_stylo.shape[1])#Nb colonnes -> caractéristiques
\end{python}
On a le même nombre de ligne, on peut donc faire une concaténation horizontale :
\begin{python}
from scipy.sparse import hstack
X_fusion = hstack((X, sparse_stylo))
\end{python}

\begin{python}
X_train, X_test, y_train, y_test = train_test_split(X_fusion, y, test_size=0.3, random_state=0)
DT = tree.DecisionTreeClassifier()
DT = DT.fit(X_train, y_train)
y_pred = DT.predict(X_test)
matrice_confusion = confusion_matrix(y_test, y_pred)
print(matrice_confusion)
stats = precision_recall_fscore_support(y_test, y_pred)
print(stats)
report = classification_report(y_test, y_pred, target_names=nom_classes)
print(report)
\end{python}
Malheureusement, ça marche moins bien qui le BOW tout seul (ce qui ne signifie pas que ce serait le cas sur tout jeu de données).

\section{Pour aller plus loin}

Dans un nouveau notebook, explorez les possibilités offertes par les forêtes d'arbres aléatoires : \small{\url{https://scikit-learn.org/stable/modules/generated/sklearn.ensemble.RandomForestClassifier.html}}


